\section{Referencia de la Clase Splash}
\label{classSplash}\index{Splash@{Splash}}
Muestra y administra la ventana de bienvenida al programa.  


{\tt \#include $<$splashscreen.h$>$}

\subsection*{Slots p\'{u}blicos}
\begin{CompactItemize}
\item 
bool {\bf event} (QEvent $\ast$)\label{classSplash_i0}

\item 
void {\bf paint} ()
\end{CompactItemize}
\subsection*{M\'{e}todos p\'{u}blicos}
\begin{CompactItemize}
\item 
{\bf Splash} ()
\end{CompactItemize}


\subsection{Descripci\'{o}n detallada}
Muestra y administra la ventana de bienvenida al programa. 



\subsection{Documentaci\'{o}n del constructor y destructor}
\index{Splash@{Splash}!Splash@{Splash}}
\index{Splash@{Splash}!Splash@{Splash}}
\subsubsection{\setlength{\rightskip}{0pt plus 5cm}Splash::Splash ()}\label{classSplash_a0}


Se modifica la paleta para que utilize la imagen como fondo.

Poniendo el minimo y maximo a 0 hace el efecto especial.

Nos muestra la ventana en modo MODAL. 

\subsection{Documentaci\'{o}n de las funciones miembro}
\index{Splash@{Splash}!paint@{paint}}
\index{paint@{paint}!Splash@{Splash}}
\subsubsection{\setlength{\rightskip}{0pt plus 5cm}void Splash::paint ()\hspace{0.3cm}{\tt  [slot]}}\label{classSplash_i1}


Cuenta el numero de mensajes.

Recorre todos los elementos del array de mensajes cada vez que se llama a la funcion. Cuando termina de recorrerlos todos cierra la ventana y el programa continua.

Asegura que los ultimos mensajes son visibles haciendo el desplazamiento necesario. 

La documentaci\'{o}n para esta clase fu\'{e} generada a partir de los siguientes archivos:\begin{CompactItemize}
\item 
splashscreen.h\item 
splashscreen.cpp\end{CompactItemize}
